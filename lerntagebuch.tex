\documentclass{scrarticle}

\begin{document}

\title{Lerntagebuch}
\author{Jakob Waibel}
\date{\today}

\maketitle 

Ich werde dieses Lerntagebuch unter anderem auch dazu nutzen, mich vor dem schreiben aufzuwärmen. 
Ich habe mich dazu entschieden das Paper iterativ zu schreiben. Ich schreibe um das Thema selber zu verstehen ein erstes Konzept mit erklärungen der verschiedenen Themen die relevant erscheinen. In der zweiten Iteration erstelle ich eine koherentere Version des bisherigen Inhalts und in der letzten Iteration überarbeite ich den Inhalt und die Formulierungen.
Die Latex-Einführung half mir dabei, Latex lieben zu lernen. Ich verwendete Latex häufig zuvor ohne wirklich zu wissen, was dahinter steckt und was ich tue. Seit der Einführung habe ich mich kontinuierlich weiter mit Latex beschäftigt bis ich zu dem Punkt gelangt bin, sogar meine Präsentation mit Hilfe von der Latex-Documentclass "beamer" zu erstellen. 
Ich habe mich dazu entschieden, die "revtex4" documentclass für mein Paper zu verwenden. Es ist ein Standard aus der Physik, welcher mich auch schon in vielen anderen Themengebieten aufgefallen ist. Durch diesen Standard ist das Paper zwar nur die hälfte der Seiten lang, sieht aber dafür so aus, wie ich mir ein Paper immer vorgestellt habe. 
Ich frage mich häufig, welches Wissen ich zitieren muss und was ich einfach so niederschreiben darf, da es keine Intellektuelle höhe benötigt hat, um diese Dinge festzustellen. 

\end{document}