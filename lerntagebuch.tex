\documentclass{scrarticle}

\begin{document}

\title{Lerntagebuch}
\author{Jakob Waibel}
\date{\today}

\maketitle 
Dieses Lerntagebuch wird unter anderem auch dazu genutzt, mich vor dem schreiben aufzuwärmen.

Ich habe mich dazu entschieden das Paper iterativ zu schreiben. Ich schreibe um das Thema selber zu verstehen ein erstes Konzept mit erklärungen der verschiedenen Themen, welche relevant erscheinen. In der zweiten Iteration erstelle ich eine koherentere Version des bisherigen Inhalts und in der letzten Iteration überarbeite ich den Inhalt und die Formulierungen.
Die Latex-Einführung half mir dabei, Latex lieben zu lernen. Ich verwendete Latex häufig für Dokumentation ohne wirklich zu wissen, was dahinter steckt und was ich tue. Seit der Einführung habe ich mich kontinuierlich weiter mit Latex beschäftigt bis ich zu dem Punkt gelangt bin, sogar meine Präsentation mit Hilfe der Latex-Documentclass "beamer" zu erstellen. 

In meinem Paper wird die "revtex4" documentclass verwendet. Es ist ein Standard aus der Physik, welcher mich auch schon in vielen anderen Themengebieten aufgefallen ist. Durch diesen Standard ist das Paper zwar nur die hälfte der Seiten lang, sieht aber dafür so aus, wie ich mir ein Paper immer vorgestellt habe. 

Ich frage mich häufig, welches Wissen ich zitieren muss und was ich einfach so niederschreiben darf, da es keine intellektuelle Höhe benötigt hat, um diese Dinge festzustellen oder abzuleiten. Allgemeinwissen muss zwar nicht zitiert werden, aber das Allgmeinwissen ist abhängig vom Stand des Lesers. Wird ein Paper von Physiker*innen an Physiker*innen geschrieben, kann man ein anderes Allgemeinwissen vorraussetzen, als wenn Ökonom*innen an andere Ökonom*innen schreiben. Dies hat es für mich oft schwer gemacht, festzustellen, was gerade zu "meinem" Allgemeinwissen gehört und was nicht.

Der Rechercheprozess verlief genauso wie in der Vorlesung angekündigt. Nachdem man einführende Paper gelesen hat, vertieft man sich weiter mit weiteren Papers zu Themen, die im ersteren Paper nicht ausreichend behandelt wurden. Dieser Prozess hat sich dann so oft wiederholt, bis genug Wissen akkumulliert wurde, um das Paper zu verfassen. In meinem Fall ist das zum Beispiel bei "Shor's Algorithm" oder den "Josephson Junctions" aufgetreten.
Für das Paper habe ich ausschließlich englische Literaturquellen verwendet, da auch das Paper auf Englisch ist.

Während des Schreibprozesses habe ich eine Vorstellung dessen erlangt, was es heißen muss wirkliche Erkenntnisse vorbringen zu müssen. Ich würde Behaupten das Thema "Quantencomputer" nur angekratzt zu haben. Eine Person, die tatsächlich Erkenntnisse hervorbringen soll, muss unfassbar viel mehr Arbeiten lesen und verstehen als ich das im Rahmen dieser Arbeit tun musste, um auf den aktuellen Stand der Technik zu kommen und diesen zu verstehen. Daraus dann noch neue Erkenntnisse abzuleiten, lässt nachvollziehen, warum eine Doktorarbeit in der Regel mehrere Jahre dauert. 

Sehr schwer gefallen ist mir die "Related Work" section. Diese kann meiner Meinung nach erst am Ende geschrieben werden, da die Struktur des Papers dann auch in die Struktur dieses Abschnittes einfließen kann. Außerdem ist es mir schwergefallen festzustellen, ob die Related Work section die Erkenntnisse enthalten soll, die für mein Paper wichtig waren oder ob eine generelle Zusammenfassung des Papers erwartet wird. Ich hoffe, dass ich diesen Abschnitt in zukünftigen Papers noch verbessern kann.

Ein witziger Seiteneffekt dieses Papers ist, dass ich erfahren habe, dass einer der im Paper erwähnten Quantencomputer in Böblingen bei IBM steht, bei welchen in im kommenden Semester mein Praxissemester absolvieren werde. Ich werde mich informieren, ob eine Besichtigung dessen möglich ist.
\end{document}