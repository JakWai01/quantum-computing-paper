\documentclass[aps,preprintnumbers,twocolumn]{revtex4}
% \documentclass[aps,preprintnumbers]{scrartcl}
\usepackage{natbib}
\usepackage{url}

\begin{document}
\title{Quantum Computing and its Implications on Digital Security}
\email{mail@jakobwaibel.com}
\author{Jakob Waibel}

\affiliation{\it Stuttgart Media University, 70569 Stuttgart}

\date{\today}

\begin{abstract}
Lorem ipsum dolor sit amet, consectetur adipiscing elit. Donec in ex ut mauris mollis dictum. Aenean tempor scelerisque lectus, eu varius ex ullamcorper ac. Nulla quis mattis tellus. Vivamus imperdiet ante sit amet tellus dapibus imperdiet. Quisque iaculis consequat cursus. Morbi in tempor neque, eu aliquet est. Pellentesque dapibus nulla id sem scelerisque eleifend. Ut tincidunt ex nec erat pharetra pretium. Vivamus bibendum quam id nunc fermentum tincidunt. Duis congue semper nulla sed porttitor. Morbi feugiat suscipit pellentesque. Nullam eget velit vel ligula suscipit convallis.
\end{abstract}

\maketitle

\begin{section}{Introduction}
    
Since Paul Benioffs and Richard Feynman's talks on quantum computing held at the first conference on the physics of computation, 
quantum computing has become a large research area. 
Even so large, that is has become more and more difficult to understand the basics of it and its implications on the future of computing. 
This work will introduce the reader to the basics of quantum computing from a computer scientists perspective. 
Hence it is important to look at the implications on digital security, especially cryptography.

\end{section}

\begin{section}{Quantum Computers}
Quantum computers utilize qubits to represent information. 
Instead of working with a stream of electiral or optical signals representing 1s and 0s, 
quantum computers utilize qubits, which can represent various combinations of 1s and 0s at the same time. 
This effect is called superposition. To put qubits into superposition, 
researchers manipulate qubits in different ways. As quantum computing is based on quantum mechanics and hence has to obey to the same laws, 
a measurement determines the state of a qubit. Once qubits are measured, 
its quantum state immediately collapses to either 1 or 0 and its state of superposition is lost.

Quantum entanglement is a phenomenon which causes particles to link so that their state cannot be described independently anymore.
This results in those entangled qubits having a similar quantum state at all times.
If the state of one of those entangled qubits changes,
the other one will instantaneously change its state as well. This improves the processing speed of quantum computers. As Alamira Jouman Hajjar puts it in her Article on AIMultiple \cite{AIMultiple}, one qubit will reveal information about all the qubits said qubit is entangled with.
This entanglement allows quantum computers to increase its computational power exponentially,
rather than linearly on regular computers.
According to an article of IEEE Spectrum \cite{IEEE}, a quantum computer equipped with 300 qubits could represent more states than there are atoms in the observable universe.

The quantum behaviour of qubits is determined to decay.
This effect is known as the scientific term "decoherence".
It describes the quantum state as being extremely fragile.
The slightest disturbance can cause a qubit to collapse and hence loose its state of superposition.
These disturbances can be caused by vibrations, change of temperatures or related external influeces.
Disturbances are also refered to as "noise" in quantum computer terminology.
Decoherence is the reason why qubits quantum computers are held in extremly protected environments, 
often fridges or vacuum chambers.
To compensate for this noise, 
thousands of standard qubits will be needed to create a reliable qubit, 
refered to as a "logical" qubit. 
A logical qubit has a long enough coherence time to be used for quantum logic gates. 
Essentially, logical qubits are a group of regular qubits refered to as a singular qubit in computation, 
which have more stable characteristics than regular qubits.

To get a sense of how far development of quantum computers has come, 
according to Siobhan Roberts in MIT Technology Review \cite{MIT}, 
the most qubits used in a quantum computer, 
as of November 17, 2021 are 256 regular qubits in a quantum computer by the Boston startup QuEra Computing. 

Since the first proposals revolving around quantum computers by Yuri Manin in 1980 and Richard Feynman in 1981,
the term "quantum supremacy" is used to denote the point at which a quantum computer can complete a mathematic calculation that is beyond reach of even the most powerful supercomputers.
It is unclear how many qubits will be needed to achieve superposition. 
This results out of the fact that research results in new algorihtms and hardware improvements to boost the performance of classical (super)computers. 

Potential applications for quantum computer exceed cryptography. Some potential applications include the simulation of quantum systems, machine learning, computational biology and generative chemistry. 

There is not a single way to build a quantum computer. 
Instead, different approaches can be followed to develop machines utilizing quantum effects for computing. Some of which will be discussed in the following section.
\end{section}

\begin{section}{Experimental Realizations of Quantum Computers}

There are different approaches to realize quantum computers.
Those different approaches have different advantantages and challenges. 
Some of those realizations, especially the ones important with regards to digital security, 
are summarized in the following:

\begin{subsection}{Ion Traps}
Vedral and Plenio describe the workings of the ion trap in their paper \cite{2008} as follows. 
They state that the ion trap quantum computer represents qubits in energy states and in vibrational modes between ions. 
In a linear ion trap, currents in electrodes generate a time dependent electric field. 
Ions then move through this potential and for some currents and ion masses, the ions get trapped in the field. 
The equilibrium is caused by the force from the electric field generated by the electrodes and the electrostatic repulsion of the ions. 
The ions form a string seperated by only a few wavelengths of light. 
The distance of a few wavelengths is sufficient to operate each ion with a different laser. 
A challenge the researchers faced when trying to realize an ion trap is the mechanical degree of freedom the ions have. 
In general, the ions are trapped at their position but they are in fact not resting but oscillating around their equilibrium position. 
The ions then get cooled by using laser cooling to reduce the movement of the ions in their current position. 
This process is able to cool down the ions near 0 Kelvin. 
At this point, the only movement left would be the movement caused by the quantum mechanical uncertainty priciple. 
So far, it is not possible to cool a whole string of ions to near 0 K, 
although several attempts cooling down a single ion and two ions succeeded. 

As mentioned by Prashant in his paper \cite{prashant}, it has been possible to evaluate Fourier transforms using the ion trap quantum computer.
This achievment lead to Shor's factoring algorithm, 
which is based on performing Fourier transforms.

The ion trap could demonstrate the fundamental operations of a quantum computer with the highest accuracy so far. 
Ion trap quantum computers are researched because they could potentially be used for a large-scale quantum computer as scaling the ion trap to a large number of qubits could be accomplished by combining multiple ion traps into an array. Ions then could be transported between the ions traps. 
As of 2018, the largest number of particles to be controllable entangled is 20 trapped ions. 
In 2021, scientists from the University of Innsbruck presented a quantum computer based on the ion trap that fits inside two 19-inch server racks.
\end{subsection}

\begin{subsection}{Nuclear Magnetic Resonance}
As explained in the article by Gershenfeld \cite{Gershenfeld1998QuantumCW}, a nuclear magnetic resonance (NMR) computer consists of a capsule filled with a liquid and an nuclear magnetic resonance spectrometer which is a machine used to analyze the molecular structure of a material by measuring the interaction of nuclear spins when placed in a strong magnetic field.
Each molecule in the liquid represents an quantum memory register. 
NMRs can compute by sending radio signals to the sample using the NMR machine and evaluating the response. 
In an NMR quantum computer, qubits are represented as spin states of the atoms in the molecules. 
In contrast to the ion trap quantum computer, 
a measurement is performed on a statistical ensemble of molecules instead of a single isolated quantum system which was used as memory register. 

Initially, as mentioned above, the NMRs used atoms in a liquid sample as qubits which is known as liquid state NMR (LSNMR). 
Since solid state NMRs (SSNMR) emerged, SSNMR is usually preferred in terms of performing quantum computation.

Solid state NMR (SSSNMR) differ from LSNMR in that SSSNMR provide a solid state sample.
This has many advantages such as lack of molecular diffusion coherence and lower temperatures, as well as the more precise localization of qubits using crystal structures.
In SSNMRs, qubits can be measured individually, same as in the ion trap,
instead of having to measure an ensemble of molecules as in LSNMR. 

NMRs have been succesfully used in quantum computing, as they can solve non-polynomial problems in a polynomial time.
According to Gershenfeld \cite{Gershenfeld1998QuantumCW}, the first successful computation on a NMR quantum computer was to execute a search using Grover's algorithm, 
which will be covered in the respective chapter. 
\end{subsection}

\begin{subsection}{Josephson Junctions}
As Martinis and Osborne explain in their paper, 
the Josephson effect is a phenomenon that occurs when two superconductors are seperated 
by some barrier or restriction between them, which is thin enough that electrons can tunnel through the barrier using quantum-mechanics. 
This tunneling produces a current, known as supercurrent, that flows continuosly without any voltage applied. 
This is used to create a non-linear inductance and design the qubits. 
Wendin and Shumeiko mention \cite{shumeiko} many ways to realize qubits in Josephson junction based quantum computers. 
The simplest of which is to realize qubits in Josepshon junctions by comparing energy levels of josepshon energy to the normal charging energy. 
The josephson energy comes from kinetic energy of the tunneling electrons.

Martinis and Osborne state \cite{martinis} that Josephson junction quantum computers are especially interesing because superconductor could provide long coherence times and because increasing the number of qubits should be straightforward. 

Qubits in Josephson junction quantum computers are controlled electrically which results in them being interesting for future developments.
\end{subsection}

\begin{section}{Challenges in Quantum Computing}
As of 2021, there are some challenges researcher in the current state of quantum computing which can be derived from the earlier chapters: 

\begin{itemize}
    \item Often, as e.g. in the Josephson junction quantum computer, extremely low temperatures are needed. These have to be achieved at reasonable cost. Using e.g. laser cooling is a one approach in quantum computing. The concept of laser cooling was proposed by Steven Chu, Claude-Cohen-Tannoudji and William D. Phillips who got awarded the Nobel prize for their research in the develoment of methods to cool and trap atoms with laser light.
    \item Quantum algorithms have probabilistic characteritics. 
    This implies that in a single operation a quantum computer returns multiple solutions, each of those solutions with an assigned probability, where only one can be correct. 
    Figuring out the correct answer reduces the potential speed of quantum computers.
    \item Qubits are prone to errors earlier introduced as noise. 
    Noise can be caused e.g. by heat or noise in the environment. 
    As in classical computing, this noise can lead to bit-flips. There are approached in "quantum error correction" to deal with these errors. 
    One of those approaches is to use the "bit flip code" in which a total of 3 qubits are needed to result in 1 correct bit using quantum entanglement.  
    \item Increasing the number of qubits in a processor. The more qubits in a processor, the more logical qubits can be used. More qubits also lead to less errors since methods for error correction can be used more effectively.
    \item Decoherence leads to qubits retaining their quantum state only for a short period of time. 
    According to Mavroeidis \cite{DBLP:journals/corr/abs-1804-00200} The longest time qubits remained in superposition so far was set by Australian researchers at the University of New South Wales. 
    Their qubits remained in superposition for a total of 35 seconds.
\end{itemize}

\end{section}

\end{section}

\begin{section}{Quantum computing algorithms and their effects on current security mechanisms}
As discussed earlier, quantum computers could eventually reach the state of quantum superiority, 
in which the quantum computer can perform tasks no regular computer could ever compute.
This implies that the quantum computer could also solve problems which a regular computer could not solve so far.
Some of the problems that can be solved and which security mechanisms are affected will be discussed in this following section. 

\begin{subsection}{Shor's algorithm}
Shor's algorithm is a quantum computer algorithm and got developed by Peter Shor in 1994.
Essentially, it can find the prime factors of an integer in the complexity class BQP,
which is defined as the class of decision problems solvable by a quantum computer in polynomial time with an error probability of at most $\frac{1}{3}$ for all instances. 
Shor's algorithm is nearly exponentially faster than the general number field sieve, which is the most efficient known classical factoring algorithm.
The efficiency of the quantum Fourier transform which, as earlier discussed, can be caluclated using e.g. an ion trap quantum computer, is responsible for the efficiency of Shor's algorithm.
The algorithm is generally capable to break current public-key cryptography mechanisms as RSA or the Diffie-Hellman key exchange if the quantum computer could operate with little noise. 
That these algorithms could be vulnerable was already stated by Kirsch in 2015
\cite{Kirsch2015QuantumCT}.

Bone and Castro \cite{Bone} describe the workings of Shor's algorithm as follows. First, a memory register is placed into a superposition of all its possible states. Think of all possible permutations of $n$ bits existing in parallel.
A calculation performed with Shor's algorithm can be thought of as computations done in parallel for every of the permutations generated before. 
They then describe the mathematics performed on the registers: 
\begin{itemize} 
    \item N describes the integer to be factorized
    \item X is a random number in range $1 < X < N-1$
    \item X is raised to one of the permutations generated earlier.
    \item The remainder of this operation is placed in a second register. The first register contains the permutation.
\end{itemize}
After performing these operations on all permutations, it is noticable that the contents of the second register repeat e.g. for $N = 15$ the repeating sequence would be $1,2,4,8,1,2,4...$.
Now the frequency of repetition can be calculated by using a complex operation on the second register and the looking at the contents which causes the results from every permutation to interfere. A possible prime factor can then be calculated by using the frequency of repetition $f$:
\begin{equation}
Factor P = X^{\frac{f}{2}}-1
\end{equation}
The inteference that provides $f$ will lead to the correct answer with a high probability as incorrect answers cancel each other out when interfering. This cannot be guaranteed! Checking that the resulting number is correct can be achieved by performing a simple multiplication. Repeating the process for different random values X has then a high chance of revealing all prime factors.

So far, the largest number factorized using Shor's algorithm, 
as of 2019 is the number 21 factored back in 2012 by Martín-López et al \cite{article}. 
The integer 35 was attempted in 2019 on an IBM Q System One, 
but the algorithm failed. 
\end{subsection}

\begin{subsection}{Grover's algorithm}
Grover's algorithm, devised by Lov Grover in 1996,
refers to a quantum algorithm for performing unstructered search. 
The analogous problem in classical computing has $O(N)$ complexity. 
Grover's algorithm is achieving the same goal in $O(\sqrt{N})$. 
In contrast to Shor's algorithm providing exponential speedup, 
Grover's algorithm only provides quadratic speedup. 

As \cite{DBLP:journals/corr/abs-1804-00200} state in their paper, Grover's algorithm can be utilized to find a collision in a hash function in $O(\sqrt{N})$ as searching for a collision is comparable to performing an unsorted search. 
Additionally, Grover's algorithm can be used to improve a lot of brute-force attacks targeting symmetric cryptography.
Brassard proved \cite{1998} that it is possible to combine Grover's algorithm with the birthday paradox, 
also known as a quantum birthday attack, which leads to finding collisions with a complexity of $O(N^{\frac{1}{3}})$.
As a result, many of current , rather weak, hashing alogithms are disqualified for use in the quantum era. 
However, as of 2021, both SHA-2 and SHA-3 with longer outputs remain quantum resistant as even Grover's algorithm does not make it feasible to to find a collision when the keys are too long.
\end{subsection}

\end{section}

% Research for this paragraph and add some more post quantum crypto examples
\begin{section}{Post Quantum Cryptography}
Post Quantum Cryptography descrbes algorithms that are secure against quantum computing and against conventional computers.
Increasing the key size from e.g. 128 bits to 256 bits squares the number of possible permutations a quantum computer has to search using Grover's algorithm.
The goal is the same as it was before quantum computers ermerged.
The operations to crack the security mechanisms have to become infeasable to perform. 

Another approach would be to develop new algorithms using more complex trapdoor functions even quantum computers could not possibly crack in a feasible amount of time. 
Researchers are working in areas like lattice-based cryptography and supersingular isogeny key exchange. 

\begin{subsection}{Lattice-based Cryptography}
This is a form of public-key cryptography that avoids the weaknesses of RSA. 
Rather than multiplying primes, lattice-based encryption schemes involve multiplying matrices. 
Furthermore, lattice-based cryptographic mechanisms are based on the presumed hardness of lattice problems. 
Here, we are given as input a lattice represented by an aribtrary basis and our goal is to output the shortest non-zero vector in it.
\end{subsection}

\end{section}

\begin{section}{Conclusion}
As described before, Shor's algorithm, 
as well as Grover's algorithm will presumably force computer scientists to improve current stadards to provide information security across areas. 
To accomplish information security, 
current security mechanisms such as hashing algorithms have to be improved and as a result be quantum-safe by increasing their key-length.
The second approach mentioned in this paper is developing new, 
quantum-secure algorithms, 
like lattice-based cryptography. 
Even the huge speedups quantum-computers provide don't make it feasible to crack these algorithms. 
It is not known yet, which type of quantum computer will prevail. 
Ion traps, Nuclear Magnetic Resonance and Josepshon Junctions are some of the most promising approaches. 
In current systems like the IBM Q System One, 
Josephson junctions are used in their superconducting quantum computer. 
If this approach will prevail is unclear, 
since quantum entanglement in this type of processor is, as of 2021, not possible yet. 
It has to be said that quantum computers, 
as what could be seen in the current records revolving around quantum computing, 
are not powerful enough to form a thread to security mechanisms by now, 
since way more qubits are needed to perform operations on large enough numbers and preserve the state of superposition long enough. 
To conclude, it should be said that quantum computing should be seen as an opportunity for overdue improvement, not as a thread to digital security. 
\end{section}

\bibliographystyle{plain}
\bibliography{citations}

\end{document}

